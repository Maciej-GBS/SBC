\documentclass[a4paper,12pt]{article}
\usepackage[spanish]{babel}
\usepackage{hyperref}
\usepackage{enumitem}
\usepackage{float}
\usepackage{multirow}

%opening
\title{Conceptualización\\ \normalsize{Clasificador de plumas}}
\author{Maciej Nalepa}

\begin{document}

\maketitle

\section{Diccionario de los conceptos}
\begin{table}[H]
	\centering
	\begin{tabular}{|l|l|p{0.3\linewidth}|l|}
		\hline
		ATRIBUTO    & SÍMBOLO & DESCRIPCIÓN                                   & VALOR           \\ \hline\hline
		Massivness  & M       & Thickness of the calamus                      & $\{1,2,3,4,5\}$ \\ \hline
		Flexibility & T       & Overall feather flexibility                   & $\{1,2,3,4,5\}$ \\ \hline
		Length      & D       & Proportion of the calamus to feather length   & $\{1,2,3\}$     \\ \hline
		Width       & S       & Proportion of maximum width to feather length & $\{1,2,3\}$     \\ \hline
	\end{tabular}
	\caption{Propiedades comunes entre tratados tipos de plumas.}
\end{table}

\begin{table}[H]
	\centering
	\begin{tabular}{|l|l|l|l|}
		\hline
		ATRIBUTO                    & SÍMBOLO            & VALOR & ALCANCE       \\ \hline\hline
		\multirow{5}{*}{Massivness} & \multirow{5}{*}{M} & 1     & $(0.5;2)$ mm  \\ \cline{3-4}
		                            &                    & 2     & $(2;5)$ mm    \\ \cline{3-4}
		                            &                    & 3     & $(5;7)$ mm    \\ \cline{3-4}
		                            &                    & 4     & $(7;9)$ mm    \\ \cline{3-4}
		                            &                    & 5     & $(9;12)$ mm   \\ \hline\hline
		\multirow{3}{*}{Length}     & \multirow{3}{*}{D} & 1     & $(0.05;0.15)$ \\ \cline{3-4}
		                            &                    & 2     & $(0.15;0.25)$ \\ \cline{3-4}
		                            &                    & 3     & $(0.25;0.35)$ \\ \hline\hline
		\multirow{3}{*}{Width}      & \multirow{3}{*}{S} & 1     & $(0.01;0.05)$ \\ \cline{3-4}
		                            &                    & 2     & $(0.05;0.2)$  \\ \cline{3-4}
		                            &                    & 3     & $(0.2;0.3)$   \\ \hline
	\end{tabular}
	\caption{El alcance de los valores posibles de los propiedades.}
\end{table}

\section{Modelo de proceso}
\subsection{Meta}
La meta del proceso es analizar y clasificar una pluma. Queremos reconocer el especie del pájaro.

\subsection{Entradas necesarias}

\subsection{Salidas producidas}
La salida producida es el tipo de pluma y el especie de cual la venga.

\subsection{Tareas}

\section{Mapa de conocimiento}

\end{document}
