\documentclass[a4paper,12pt]{article}
\usepackage[spanish]{babel}
\usepackage{hyperref}
\usepackage{enumitem}
\usepackage{float}
\usepackage{multirow}

%opening
\title{Conceptualización\\ \normalsize{Clasificador de plumas}}
\author{Maciej Nalepa}

\begin{document}

\maketitle

\section{Dominio de aplicación}
El sistema va a tratar los siguientes grupos de pájaros:
\begin{itemize}
	\item Anseriformes (Anatidae)
	\item Suliformes (Phalacrocoracidae)
	\item Podicipediformes (Podicipedidae)
	\item Pelecaniformes (Threskiornithidae)
	\item Ciconiiformes (Ciconiidae)
	\item Gruiformes (Gruidae)
	\item Phoenicopteriformes (Phoenicopteridae)
	\item Pelecaniformes (Ardeidae)
	\item Columbiformes
	\item Accipitriformes
	\item Strigiformes
	\item Passeriformes (Corvidae)
\end{itemize}

\section{Diccionario de los conceptos}
\begin{table}[H]
	\centering
	\begin{tabular}{|p{0.2\linewidth}|p{0.15\linewidth}|p{0.25\linewidth}|p{0.4\linewidth}|}
		\hline
		ATRIBUTO       & SÍMBOLO      & DESCRIPCIÓN                                   & VALOR                                                                                                                                                                                                                           \\ \hline\hline
		Massivness     & M            & Thickness of the calamus                      & $(0.5;12)$ mm                                                                                                                                                                                                                 \\ \hline
		Flexibility    & T            & Overall feather flexibility (where 5 is least flexible)             & $\{1,2,3,4,5\}$                                                                                                                                                                                                                 \\ \hline
		Length         & D            & Proportion of the calamus to feather length   & $(0.05;0.35)$                                                                                                                                                                                                                     \\ \hline
		Width          & S            & Proportion of maximum width to feather length & $(0.01;0.3)$                                                                                                                                                                                                                     \\ \hline
		Feather length & length       & The length of the feather (straightened)      & $(35;570)$ mm                                                                                                                                                                                                                   \\ \hline
		Calamus        & calamus      & Colours.                                      & \small \{'brown', 'darkgrey', 'grey', 'lightgrey', 'white', 'yellow'\}                                                                                                                                                                 \\ \hline
		Inner vane     & vane-inner   & Colours and characteristics.                  & \small \{'black', 'brown', 'darkbrown', 'darkgrey', 'glossy', 'grey', 'gris', 'lightbrown', 'lightgrey', 'lightgris', 'lightorange', 'lightyellow', 'orange', 'white', 'yellow'\} \\ \hline
		Outer vane     & vane-outer   & Colours and characteristics.                  & \small \{'black', 'brown', 'darkbrown', 'darkgrey', 'glossy', 'grey', 'gris', 'lightbrown', 'lightgrey', 'lightorange', 'lightyellow', 'orange', 'pink', 'white', 'yellow'\}      \\ \hline
		Rachis (lower) & rachis-lower & Colours.                                      & \small \{'black', 'brown', 'darkgrey', 'grey', 'lightgrey', 'white'\}                                                                                                                                                       \\ \hline
		Rachis (upper) & rachis-upper & Colours and characteristics.                  & \small \{'black', 'brown', 'darkbrown', 'darkgrey', 'grey', 'lightbrown', 'lightgrey', 'white'\}                                                                                                                            \\ \hline
		Upper vanes    & vanes-upper  & Colours and characteristics.                  & \small \{'black', 'brown', 'darkbrown', 'darkgrey', 'grey', 'lightbrown', 'lightgrey', 'lightgris', 'lightyellow', 'white'\} \\ \hline
		Stripes & stripes & Feather striping. & \{'visible', 'weak', 'wide'\} \\ \hline
		Rachis stripes & rachis-striped & Flag whether rachis is striped. & Existent or not \\ \hline
		Gloss & glossy & Feather vanes glossiness. & Existent or not \\ \hline
	\end{tabular}
	\caption{Propiedades comunes entre tratados tipos de plumas.}
\end{table}

\begin{table}[H]
	\centering
	\begin{tabular}{|p{0.2\linewidth}|p{0.4\linewidth}|p{0.3\linewidth}|}
		\hline
		SÍMBOLO & DESCRIPCIÓN       & VALOR \\ \hline\hline
		type-length       & Feather types matching given feather length. & feather types  \\ \hline
		species-length       & Species matching given feather length. & species names  \\ \hline
		species-glossy       & Species matching given glossiness. & species names  \\ \hline
		species-rachis-striped       & Species matching rachis striping. & species names  \\ \hline
		type-stripes       & Feather types matching striping given. & feather types  \\ \hline
		species-stripes       & Species matching vane striping given. & species names  \\ \hline
	\end{tabular}
\caption{Propiedades intermedios.}
\end{table}

\begin{table}[H]
	\centering
	\begin{tabular}{|l|l|l|l|}
		\hline
		ATRIBUTO                    & SÍMBOLO            & VALOR & RANGO       \\ \hline\hline
		\multirow{5}{*}{Massivness} & \multirow{5}{*}{M} & 1     & $(0.5;2)$ mm  \\ \cline{3-4}
		                            &                    & 2     & $(2;5)$ mm    \\ \cline{3-4}
		                            &                    & 3     & $(5;7)$ mm    \\ \cline{3-4}
		                            &                    & 4     & $(7;9)$ mm    \\ \cline{3-4}
		                            &                    & 5     & $(9;12)$ mm   \\ \hline\hline
		\multirow{3}{*}{Length}     & \multirow{3}{*}{D} & 1     & $(0.05;0.15)$ \\ \cline{3-4}
		                            &                    & 2     & $(0.15;0.25)$ \\ \cline{3-4}
		                            &                    & 3     & $(0.25;0.35)$ \\ \hline\hline
		\multirow{3}{*}{Width}      & \multirow{3}{*}{S} & 1     & $(0.01;0.05)$ \\ \cline{3-4}
		                            &                    & 2     & $(0.05;0.2)$  \\ \cline{3-4}
		                            &                    & 3     & $(0.2;0.3)$   \\ \hline
	\end{tabular}
	\caption{El diccionario de escalas sobre las propiedades.}
\end{table}

\section{Modelo de proceso}
\subsection{Meta}
La meta del proceso es analizar y clasificar una pluma. Queremos reconocer el especie del pájaro.

\subsection{Entradas necesarias}
Las siguientes entradas no se puede inferir, sino a veces no todos son necesarias para clasificar bien.
\begin{itemize}
	\item M -- Massivness
	\item T -- Flexibility
	\item D -- Length
	\item S -- Width
	\item length -- Feather length
	\item calamus -- Calamus
	\item vane-inner -- Inner vane
	\item vane-outer -- Outer vane
	\item rachis-lower -- Rachis (lower)
	\item rachis-upper -- Rachis (upper)
	\item vanes-upper -- Upper vanes
\end{itemize}

\subsection{Salidas producidas}
La salida producida es el tipo de pluma y el especie de cual la venga.

\subsection{Tareas}
\begin{enumerate}
	\item Buscar las características cumplidas
	\item Reconocer el tipo de la pluma
	\item Reconocer el grupo de las especies
	\item Encontrar la especie cual pertenece al grupo
\end{enumerate}

\end{document}
