\documentclass[a4paper,12pt]{article}
\usepackage[spanish]{babel}
\usepackage{hyperref}
\usepackage{enumitem}
\usepackage{float}
\usepackage{multirow}

%opening
\title{Conceptualización\\ \normalsize{Clasificador de plumas}}
\author{Maciej Nalepa}

\begin{document}

\maketitle

\section{Diccionario de los conceptos}
\begin{table}[H]
	\centering
	\begin{tabular}{|l|l|p{0.3\linewidth}|p{0.3\linewidth}|}
		\hline
		ATRIBUTO                 & SÍMBOLO & DESCRIPCIÓN                                   & VALOR           \\ \hline\hline
		Massivness               & M       & Thickness of the calamus                      & $\{1,2,3,4,5\}$ \\ \hline
		Flexibility              & T       & Overall feather flexibility                   & $\{1,2,3,4,5\}$ \\ \hline
		Length                   & D       & Proportion of the calamus to feather length   & $\{1,2,3\}$     \\ \hline
		Width                    & S       & Proportion of maximum width to feather length & $\{1,2,3\}$     \\ \hline
		Feather length           & FL      & The length of the feather (straightened)      & $(50;570) mm$   \\ \hline
		Colour of calamus        & CC      & Self explanatory.                             & $()$            \\ \hline
		Colour of inner vane     & CIV     & Self explanatory.                             & $()$            \\ \hline
		Colour of outer vane     & COV     & Self explanatory.                             & $()$            \\ \hline
		Colour of rachis (under) & CRL     & Self explanatory.                             & $()$            \\ \hline
		Colour of rachis (upper) & CRU     & Self explanatory.                             & $()$            \\ \hline
		Colour of upper vanes    & CRUP    & Self explanatory.                             & \{'black', 'white lightgrey', 'lightgrey white', 'grey black white', 'lightgrey', 'white black brown stripes', 'white black', 'white stripes', 'white brown stripes', 'grey brown', 'brown grey', 'black brown', 'black brown stripes', 'grey black', 'grey', 'lightgrey darkgrey', 'lightgrey grey darkgrey black ragged', 'white', 'white grey', 'black brown ~stripes', 'grey white'\}            \\ \hline
	\end{tabular}
	\caption{Propiedades comunes entre tratados tipos de plumas.}
\end{table}

\begin{table}[H]
	\centering
	\begin{tabular}{|l|l|l|l|}
		\hline
		ATRIBUTO                    & SÍMBOLO            & VALOR & ALCANCE       \\ \hline\hline
		\multirow{5}{*}{Massivness} & \multirow{5}{*}{M} & 1     & $(0.5;2)$ mm  \\ \cline{3-4}
		                            &                    & 2     & $(2;5)$ mm    \\ \cline{3-4}
		                            &                    & 3     & $(5;7)$ mm    \\ \cline{3-4}
		                            &                    & 4     & $(7;9)$ mm    \\ \cline{3-4}
		                            &                    & 5     & $(9;12)$ mm   \\ \hline\hline
		\multirow{3}{*}{Length}     & \multirow{3}{*}{D} & 1     & $(0.05;0.15)$ \\ \cline{3-4}
		                            &                    & 2     & $(0.15;0.25)$ \\ \cline{3-4}
		                            &                    & 3     & $(0.25;0.35)$ \\ \hline\hline
		\multirow{3}{*}{Width}      & \multirow{3}{*}{S} & 1     & $(0.01;0.05)$ \\ \cline{3-4}
		                            &                    & 2     & $(0.05;0.2)$  \\ \cline{3-4}
		                            &                    & 3     & $(0.2;0.3)$   \\ \hline
	\end{tabular}
	\caption{El diccionario de escalas sobre las propiedades.}
\end{table}

\section{Modelo de proceso}
\subsection{Meta}
La meta del proceso es analizar y clasificar una pluma. Queremos reconocer el especie del pájaro.

\subsection{Entradas necesarias}

\subsection{Salidas producidas}
La salida producida es el tipo de pluma y el especie de cual la venga.

\subsection{Tareas}

\section{Mapa de conocimiento}

\end{document}
