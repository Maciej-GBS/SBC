\documentclass[a4paper,12pt]{article}
\usepackage[spanish]{babel}
\usepackage{hyperref}
\usepackage{enumitem}
\usepackage{float}
\usepackage{multirow}

%opening
\title{Conceptualización\\ \normalsize{Clasificador de plumas}}
\author{Maciej Nalepa}

\begin{document}

\maketitle

\section{Diccionario de los conceptos}
\begin{table}[H]
	\centering
	\begin{tabular}{|p{0.2\linewidth}|p{0.15\linewidth}|p{0.25\linewidth}|p{0.4\linewidth}|}
		\hline
		ATRIBUTO                 & SÍMBOLO & DESCRIPCIÓN                                   & VALOR                                                                                                                                  \\ \hline\hline
		Massivness               & M       & Thickness of the calamus                      & $\{1,2,3,4,5\}$                                                                                                                        \\ \hline
		Flexibility              & T       & Overall feather flexibility                   & $\{1,2,3,4,5\}$                                                                                                                        \\ \hline
		Length                   & D       & Proportion of the calamus to feather length   & $\{1,2,3\}$                                                                                                                            \\ \hline
		Width                    & S       & Proportion of maximum width to feather length & $\{1,2,3\}$                                                                                                                            \\ \hline
		Feather length           & FL      & The length of the feather (straightened)      & $(50;570) mm$                                                                                                                          \\ \hline
		Colour of calamus        & CC      & Self explanatory.                             & \{'grey', 'lightgrey', 'brown', 'yellow', 'darkgrey', 'white'\}                                                                        \\ \hline
		Colour of inner vane     & CIV     & Self explanatory.                             & \{'black', 'grey', 'brown', 'ragged', 'darkgrey', 'white', 'glossy', 'lightgrey', 'stripes', 'darkbrown'\}                             \\ \hline
		Colour of outer vane     & COV     & Self explanatory.                             & \{'black', 'grey', 'brown', 'yellow', 'ragged', 'darkgrey', 'white', 'glossy', 'orange', 'lightgrey', 'stripes', 'pink', 'darkbrown'\} \\ \hline
		Colour of rachis (under) & CRL     & Self explanatory.                             & \{'black', 'grey', 'lightgrey', 'brown', 'white'\}                                                                                     \\ \hline
		Colour of rachis (upper) & CRU     & Self explanatory.                             & \{'black', 'grey', 'lightgrey', 'brown', 'darkgrey', 'white'\}                                                                         \\ \hline
		Colour of upper vanes    & CRUP    & Self explanatory.                             & \{'black', 'grey', 'brown', 'ragged', 'darkgrey', 'white', 'lightgrey', 'stripes'\}                                                    \\ \hline
%		C & C & Self explanatory. & \{\} \\ \hline
	\end{tabular}
	\caption{Propiedades comunes entre tratados tipos de plumas.}
\end{table}

\begin{table}[H]
	\centering
	\begin{tabular}{|l|l|l|l|}
		\hline
		ATRIBUTO                    & SÍMBOLO            & VALOR & ALCANCE       \\ \hline\hline
		\multirow{5}{*}{Massivness} & \multirow{5}{*}{M} & 1     & $(0.5;2)$ mm  \\ \cline{3-4}
		                            &                    & 2     & $(2;5)$ mm    \\ \cline{3-4}
		                            &                    & 3     & $(5;7)$ mm    \\ \cline{3-4}
		                            &                    & 4     & $(7;9)$ mm    \\ \cline{3-4}
		                            &                    & 5     & $(9;12)$ mm   \\ \hline\hline
		\multirow{3}{*}{Length}     & \multirow{3}{*}{D} & 1     & $(0.05;0.15)$ \\ \cline{3-4}
		                            &                    & 2     & $(0.15;0.25)$ \\ \cline{3-4}
		                            &                    & 3     & $(0.25;0.35)$ \\ \hline\hline
		\multirow{3}{*}{Width}      & \multirow{3}{*}{S} & 1     & $(0.01;0.05)$ \\ \cline{3-4}
		                            &                    & 2     & $(0.05;0.2)$  \\ \cline{3-4}
		                            &                    & 3     & $(0.2;0.3)$   \\ \hline
	\end{tabular}
	\caption{El diccionario de escalas sobre las propiedades.}
\end{table}

\section{Modelo de proceso}
\subsection{Meta}
La meta del proceso es analizar y clasificar una pluma. Queremos reconocer el especie del pájaro.

\subsection{Entradas necesarias}
Las siguientes entradas no se puede inferir, sino a veces no todos son necesarias para clasificar bien.
\begin{itemize}
	\item M -- Massivness
	\item T -- Flexibility
	\item D -- Length
	\item S -- Width
	\item FL -- Feather length
	\item CC -- Colour of calamus
	\item CIV -- Colour of inner vane
	\item COV -- Colour of outer vane
	\item CRL -- Colour of rachis (under)
	\item CRU -- Colour of rachis (upper)
	\item CRUP -- Colour of upper vanes
\end{itemize}

\subsection{Salidas producidas}
La salida producida es el tipo de pluma y el especie de cual la venga.

\subsection{Tareas}

\section{Mapa de conocimiento}

\end{document}
