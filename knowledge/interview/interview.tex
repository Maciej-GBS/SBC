\documentclass[a4paper,12pt]{article}
\usepackage[spanish]{babel}
\usepackage{hyperref}
\usepackage{enumitem}
\usepackage{float}

%opening
\title{Entrevista documentada\\ \normalsize{Clasificador de plumas}}
\author{Maciej Nalepa}
\date{21 de marzo de 2021}

\begin{document}

\maketitle

\begin{enumerate}
	\item Fecha: 21/03/21
	\item Hora: 16:00 - 20:00
	\item Lugar: en línea
	\item Asistentes: Ewa Nalepa (Experto)
\end{enumerate}

\section{Situación del análisis respecto al modelo general}
Esta entrevista es dedicada a descubrir los elementos y las relaciones en el dominio de clasificación de plumas de pájaros.

\section{Conocimiento anterior a la entrevista}
El punto de partida de esta entrevista es la descripción incompleta del dominio.
Ya se sabe el alcance del proyecto y tipos de plumas tratados en el sistema.

\subsection{Lista de elementos}
\begin{itemize}
	\item[1wg] Ala
	\item[2tl] Cola
	\item[3rm] Rémiges
	\item[4p] Rémiges primarias (P)
	\item[5s] Rémiges secundarias (S)
	\item[6t] Rémiges terciarias (T)
	\item[7rc] Plumas de la cola (rectrices)
\end{itemize}

\subsection{Relaciones entre elementos}
\begin{table}[H]
	\centering
	\begin{tabular}{ll}
		4p es una parte de 1wg	& Rémiges primarias son una parte de ala	\\
		5s es una parte de 1wg	& Rémiges secundarias son una parte de ala	\\
		6t es una parte de 1wg	& Rémiges terciarias son una parte de ala	\\
		7rc es una parte de 2tl	& Rectrices son una parte de cola	\\
		4p es un subconjunto de 3rm	& Rémiges primarias es un subconjunto de rémiges	\\
		5s es un subconjunto de 3rm	& Rémiges primarias es un subconjunto de rémiges	\\
		6t es un subconjunto de 3rm	& Rémiges primarias es un subconjunto de rémiges	\\
	\end{tabular}
\end{table}

\subsection{Estado de elementos}

\section{Objetivos de la entrevista}
\begin{enumerate}[label=\Alph*]
	\item Identificar el dominio físico y relaciones entre los elementos del mismo en cual el operador ha descrito el caso anterior.
	\item Completar el conocimiento de modo sobre los desencadenantes de actuaciones del operador en el caso conreto descrito en la entrevista anterior.
\end{enumerate}

Fuentes de conocimiento: Experto

Modo: entrevista parcialmente estructurada

\section{Planteamiento de la sesión}
\begin{itemize}
	\item[A1] Z czego składa się pióro w celu klasyfikacji?
	\item[A3] Jakie pomiary są pobierane w celu klasyfikacji?
	\item[A4] Jakie kolory pióra pomagają w klasyfikacji?
	\item[A5] Kolory których elementów pióra są istotne w klasyfikacji?
	\item[A6] Czy są jeszcze inne istotne elementy, które pozwalają sklasyfikować pióro?
	\item[A7] Czy płeć ma znaczenie?
	\item[A8] Czy wiek ptaka ma znaczenie?
	\item[B1] Najpierw należy rozpoznać typ pióra, czy gatunek?
	\item[B2] W jaki sposób rozpoznajemy typ pióra?
	\item[B3] W jaki sposób zawęża się grupę gatunków, do których może należeć pióro?
\end{itemize}

\section{Resultado de la sesión}
\begin{itemize}
	\item[A1] oś pióra = dudka+stosina, chorągiewki - zewnętrzna i wewnętrzna, puch przynasadowy (obecny / nieobecny)
	\item[A3] długość osi pióra - tj całkowita długość wyprostowanego pióra
	długość dudki - tj stosunek dł dudki do całkowitej długości pióra
	\item[A4] Kolor podstawowy tła chorągiewek (1 do 3) - spod, zewnetrzna, wewnetrzna
	Kolor stosiny (wierzch, spod) i dudki (1)
	\item[A5] Kolor chorągiewki zewnętrznej
	Kolor chorągiewki wewnętrznej
	Kolor puchu (jeśli obecny)
	Kolor stosiny i dudki z wierzchu
	Kolor stosiny i dudki od spodu
	Spód obu chorągiewek
	\item[A6] Zapach - jeśli intensywny i bardzo mocny, to cecha typowa dla kormorana, oraz kruka (choć kormoran cuchnie bardziej xd)
	Ogólnie na co patrzeć oprócz rodzaju pióra, jego długości i długości dudki. Warto patrzeć na masywność / delikatność dudki i stosiny, masywność / delikatność chorągiewek w dotyku, ew ich półprzezroczystość, obecność puchu przynasadowego i jego kolor, a także ogólną elastyczność / twardość / miękkość całego pióra, tak chorągiewek, jak i osi. Więcej info w opisach gatunków. Chyba tyle xd
	\item[A7] Tak, ale zależy od gatunku.
	\item[A8] Tak, ale zależy od gatunku (będzie zaznaczone przy konkretnych gatunkach)
	\item[B1] Łoczywiście, że typ pióra! Dla przykładu, kruk i kawka mają pióra tak samo czarne i takiej samej struktury w dotyku. I dla przykładu mamy 10 cm pióra. Jeśli to lotka II rz, no to kawka, jeśli to pokrywa II rz duża, no to kruk
	\item[B2] Patrzymy na kształt stosiny i kształt wierzchołka pióra i długość dudki (są pewne różnice międzygatunkowe, ale w obrębie jednego osobnika schemat się powtarza)
	\item[B3] Zaczynamy od cech charakterystycznych. To na podstawie tych ogólnych cech, które są najlepiej opisane w tabeli. Patrzysz na ogólny charakter pióra i przypasowujesz go do grupy, jednej z tych 15 czy ilu tam mi wyszło. To chyba jedyny sposób, tu już trzeba niestety te cechy po kolei sprawdzić, jak ogólnie pióro się miewa, a potem się człowiek wdraża w szczegóły by rozpoznać konkretne gatunki. Przydają się też te cechy charakterystyczne w przypadku wybranych grup, bo pozwalają to mega szybko zawęzić :P ogólnie to temat dotyczy klucza do id piór, za którego tworzenie powoli się zabrałam ostatnio i rozmknia jest niezła, mam za dużo expa i dla mnie wszystkie pióra są już zbyt różne, choć wiem,  że “teoretycznie wiele z nich” można by wsadzić do mniejszego lub większego worka xp No… generalnie pacz do tabeli
\end{itemize}

\section{Plan de análisis}
\begin{enumerate}
	\item Identificación de términos
	\item Identificación del dominio de elementos
	\item Generación de glosario
	\item Identificación de relaciones entre elementos
\end{enumerate}
Los términos son características físicas de plumas.

\section{Resultados del análisis}

\end{document}
