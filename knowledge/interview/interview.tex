\documentclass[a4paper,12pt]{article}
\usepackage[spanish]{babel}
\usepackage{hyperref}
\usepackage{enumitem}
\usepackage{float}

%opening
\title{Entrevista documentada\\ \normalsize{Clasificador de plumas}}
\author{Maciej Nalepa}
\date{21 de marzo de 2021}

\begin{document}

\maketitle

\begin{enumerate}
	\item Fecha: 21/03/21
	\item Hora: 16:00 - 20:00
	\item Lugar: en línea
	\item Asistentes: Ewa Nalepa (Experto)
\end{enumerate}

\section{Situación del análisis respecto al modelo general}
Esta entrevista es dedicada a descubrir los elementos y las relaciones en el dominio de clasificación de plumas de pájaros.

\section{Conocimiento anterior a la entrevista}
El punto de partida de esta entrevista es la descripción incompleta del dominio.
Ya se sabe el alcance del proyecto y tipos de plumas tratados en el sistema.

\subsection{Lista de elementos}
\begin{itemize}
	\item[0pl] Pluma
	\item[1wg] Ala
	\item[2tl] Cola
	\item[3rm] Rémiges
	\item[4p] Rémiges primarias (P)
	\item[5s] Rémiges secundarias (S)
	\item[6t] Rémiges terciarias (T)
	\item[7rc] Plumas de la cola (rectrices)
\end{itemize}

\subsection{Relaciones entre elementos}
\begin{table}[H]
	\centering
	\begin{tabular}{ll}
		1wg y 2tl consiste de 0pl			  & La ala y cola tienen plumas \\
		3rm y 7rc son 0pl					  & Las rémiges y las rectrices son plumas \\
		4p, 5s y 6t son un subconjunto de 3rm & P, S y T son rémiges            \\
		3rm es una parte de 1wg               & Rémiges son una parte de ala    \\
		7rc es una parte de 2tl               & Rectrices son una parte de cola
	\end{tabular}
\end{table}

\subsection{Estado de elementos}

\section{Objetivos de la entrevista}
\begin{itemize}
	\item[A] Identificar el dominio físico y relaciones entre los elementos del mismo.
	\item[B] Completar el conocimiento de modo sobre el proceso de identificación de una pluma.
\end{itemize}

Fuentes de conocimiento: Experto

Modo: entrevista parcialmente estructurada

\section{Planteamiento de la sesión}
\begin{itemize}
	\item[A1] ¿Qué son los elementos de una pluma?
	\item[A2] ¿Qué mediciones se hace?
	\item[A3] ¿Qué colores y de cuál partes de plumas confirman la clasificación?
	\item[A4] ¿Hay otras características para clasificar una pluma?
	\item[A5] ¿Es el sexo del pájaro importante?
	\item[A6] ¿Es la edad del pájaro importante?
	\item[B1] ¿Hay que reconocer el tipo de la pluma antes la especie?
	\item[B2] ¿Qué es el proceso de clasificación de una pluma?
\end{itemize}

\section{Resultado de la sesión}
\begin{itemize}
	\item[A1] Los más importantes y significantes: hay un eje de la pluma cual consiste de un cálamo y un raquis, además hay los estandartes - interno y externo.
	\item[A2] Se mide la longitud del eje, la longitud del cálamo. Exactamente nos importa la proporción del cálamo al eje.
	\item[A3] Los colores significantes sean el color básico de los estandartes (al frente y la espalda) - pueden tener de 1 hasta 3 diferentes, el color del cálamo y los colores de la raquis de ambos lados.
	\item[A4] Sí, el olor por ejemplo es muy característico a algunos pájaros, las durezas de los estandartes, del cálamo y el factor elástico del eje.
	\item[A5] Sí, pero depende de la especie.
	\item[A6] Sí, depende de la especie.
	\item[B1] Se empieza de reconocer el tipo de la pluma. Por ejemplo es la única solución para distinguir entre una pluma de un cuervo grande y de una grajilla.
	\item[B2] Al principio podemos buscar las características típicas de un grupo de pájaros para que es más fácil encontrar la clasificación. Luego reconocemos los colores de los estandartes, la forma del cima de pluma y la longitud de cálamo.
\end{itemize}

\section{Plan de análisis}
\begin{enumerate}
	\item Identificación de términos
	\item Generación de glosario
	\item Identificación de relaciones entre elementos
\end{enumerate}
Los términos son características físicas de plumas.

\section{Resultados del análisis}
\subsection{Glosario de los elementos nuevos}
\begin{itemize}
	\item[8rq] Raquis
	\item[9cm] Cálamo
	\item[10ee] Estandarte externo
	\item[11ei] Estandarte interno
	\item[12es] Estandarte
	\item[13ej] Eje de la pluma
	\item[14lg] Longitud
	\item[15cl] Color
	\item[16dr] Dureza
\end{itemize}

\subsection{Relaciones encontrados}
\begin{table}[H]
	\centering
	\begin{tabular}{ll}
		8rq es una parte de 13ej	& El raquis es la parte de arriba del eje	\\
		9cm es una parte de 13ej	& El cálamo es la parte bajo del eje \\
		10ee y 11ei son las partes de 12es	& El estandarte externo y el interno son estandartes \\
		13ej, 12es son las partes de 0pl	& El eje y los estandartes son las partes de pluma \\
		13ej, 8rq y 9cm tienen 14lg	& El eje, el raquis y el cálamo tienen un longitud \\
		13ej tiene dos 15cl	& El eje tiene dos colores \\
		12es puede tener hasta tres 15cl	& El estandarte puede tener hasta tres colores \\
		13ej y 12es tienen 16dr	& El eje y el estandarte tienen una dureza
	\end{tabular}
\end{table}

\end{document}
