\documentclass[a4paper,12pt]{article}
\usepackage[spanish]{babel}

%opening
\title{Propuesta de un sistema basado en conocimiento\\ \normalsize{Tutor de portafolio}}
\author{Maciej Nalepa}

\begin{document}

\maketitle

\section{Descripción}
Inversión de dinero es una tarea difícil cual requiere experiencia y práctica para que sea realizada correctamente.

Este sistema propuesto proporcionará el conocimiento para realizar una selección de un portafolio de valores disponible en la bolsa o algunos fondos de inversión. La meta del sistema es ofrecer consejos sobre la repartición de dinero entre múltiplas formas de inversión e elegir las empresas o acciones comunes cuales llevan el riesgo adecuado a las preferencias del usuario.

El entorno de la bolsa lleva mucho riesgo y sin experiencia se puede perder dinero invertido. Por eso existen empresas y bancos cual ofrecen los servicios de manejar dinero en una forma controlada por expertos.

\section{Alcance y limites}
El sistema va a necesitar la información sobre el estado actual de la bolsa, cual puede ser ampliado antes. Para simplificar la tarea de obtener los datos, se supone que no incluyan todos los valores de la bolsa.

El limite será proporcionar las reglas generales para agrupar las inversiones en grupos de bajo o alto riesgo y de baja o alta tasa de interés, que ayuden crear un portafolio adecuado.

\section{Viabilidad}

\noindent Hay cuatro dimensiones de viabilidad del sistema:
\begin{enumerate}
	\item Plausibilidad
	\item Justificación
	\item Adecuación
	\item Éxito
\end{enumerate}

\noindent Cada dimensión consiste de las características de tres categorías:
\begin{enumerate}
	\item Directivos y usuarios (DU)
	\item Los expertos (EX)
	\item La tarea (TA)
\end{enumerate}

\noindent El valor del umbral se establece $V_u = 7$.

\begin{table}[h]
	\centering
	\begin{tabular}{|l|l|l|}
		\hline
		Dimensión	   & Símbolo & Valor 	 \\ \hline
		Plausibilidad  & $VC_1$  & $75.20$   \\
		Justificación  & $VC_2$  & $83.66$   \\
		Adecuación     & $VC_3$  & $80.96$   \\
		Éxito          & $VC_4$  & $60.47$   \\ \hline
		\textit{Total} & $VC$    & $75.07$   \\ \hline
	\end{tabular}
	\caption{Valor global $VC$}
	\label{tab:total}
\end{table}

\begin{table}[h]
	\centering
	\begin{tabular}{|l|l|l|l|p{4cm}|l|}
		\hline
		\scriptsize CATEGORÍA & \scriptsize IDENTIFICADOR & \scriptsize PESO & \scriptsize VALOR & \scriptsize DENOMINACIÓN DE LA CARACTERÍSTICA                                           & \scriptsize TIPO \\ \hline
		EX                    & P1                        & 10               & 10                & Existen expertos                                                                        & E                \\ \hline
		EX                    & P2                        & 10               & 9                 & El experto asignado es genuino                                                          & E                \\ \hline
		EX                    & P3                        & 8                & 9                 & El experto es cooperativo                                                               & D                \\ \hline
		EX                    & P4                        & 7                & 9                 & El experto es capaz de articular sus métodos pero no categoriza.                        & D                \\ \hline
		TA                    & P5                        & 10               & 10                & Existen suficientes casos de prueba; normales, típicos, ejemplares, correosos, etc      & E                \\ \hline
		TA                    & P6                        & 10               & 10                & La tarea está bien estructurada y se entiende                                           & D                \\ \hline
		TA                    & P7                        & 10               & 10                & Sólo requiere habilidad cognoscitiva (no pericia física)                                & D                \\ \hline
		TA                    & P8                        & 9                & 10                & No se precisan resultados óptimos sino sólo Satisfactorios, sin comprometer el proyecto & D                \\ \hline
		TA                    & P9                        & 9                & 5                 & La tarea no requiere sentido común                                                      & D                \\ \hline
		DU                    & P10                       & 7                & 5                 & Los directivos están verdaderamente comprometidos con el proyecto                       & D \\ \hline
	\end{tabular}
	\caption{Plausibilidad}
	\label{tab:plausibilidad}
\end{table}

\begin{table}[h]
	\centering
	\begin{tabular}{|l|l|l|l|p{4cm}|l|}
		\hline
		\scriptsize CATEGORÍA & \scriptsize IDENTIFICADOR & \scriptsize PESO & \scriptsize VALOR & \scriptsize DENOMINACIÓN DE LA CARACTERÍSTICA                                 & \scriptsize TIPO \\ \hline
		EX                    & J1                        & 10               & 10                & El experto NO está disponible                                                 & E                \\ \hline
		EX                    & J2                        & 10               & 10                & Hay escasez de experiencia humana                                             & D                \\ \hline
		TA                    & J3                        & 8                & 10                & Existe necesidad de experiencia simultánea en muchos lugares                  & D                \\ \hline
		TA                    & J4                        & 10               & 10                & Necesidad de experiencia en entornos hostiles, penosos y/o poco gratificantes & E                \\ \hline
		TA                    & J5                        & 8                & 8                 & No existen soluciones alternativas admisibles                                 & E                \\ \hline
		DU                    & J6                        & 7                & 10                & Se espera una alta tasa de recuperación de la inversión                       & D                \\ \hline
		DU                    & J7                        & 8                & 10                & Resuelve una tarea útil y necesaria                                           & E                \\ \hline
	\end{tabular}
	\caption{Justificación}
	\label{tab:justificacion}
\end{table}

\begin{table}[h]
	\centering
	\begin{tabular}{|l|l|l|l|p{4cm}|l|}
		\hline
		\scriptsize CATEGORÍA & \scriptsize IDENTIFICADOR & \scriptsize PESO & \scriptsize VALOR & \scriptsize DENOMINACIÓN DE LA CARACTERÍSTICA                                                                                       & \scriptsize TIPO \\ \hline
		EX                    & A1                        & 5                & 9                 & La experiencia del experto está poco organizada                                                                                     & D                \\ \hline
		TA                    & A2                        & 6                & 10                & Tiene valor práctico                                                                                                                & D                \\ \hline
		TA                    & A3                        & 7                & 8                 & Es una tarea más táctica que estratégica                                                                                            & D                \\ \hline
		TA                    & A4                        & 7                & 10                & La tarea da soluciones que sirvan a necesidades a largo plazo                                                                       & E                \\ \hline
		TA                    & A5                        & 5                & 10                & La tarea no es demasiado fácil, pero es de conocimiento intensivo, tanto propio del dominio, como de manipulación de la información & D                \\ \hline
		TA                    & A6                        & 6                & 8                 & Es de tamaño manejable, y/o es posible un enfoque gradual y/o, una descomposición en subtareas independientes                       & D                \\ \hline
		EX                    & A7                        & 7                & 10                & La transferencia de experiencia entre humanos es factible (experto a aprendiz)                                                      & E                \\ \hline
		TA                    & A8                        & 6                & 8                 & Estaba identificada como un problema en el área y los efectos de la introducción de un SE pueden planificarse                       & D                \\ \hline
		TA                    & A9                        & 9                & 10                & No requiere respuestas en tiempo real "inmediato“                                                                                   & E                \\ \hline
		TA                    & A10                       & 9                & 9                 & La tarea no requiere investigación básica                                                                                           & E                \\ \hline
		TA                    & A11                       & 5                & 7                 & El experto usa básicamente razonamiento simbólico que implica factores subjetivos                                                   & D                \\ \hline
		TA                    & A12                       & 5                & 9                 & Es esencialmente de tipo heurístico                                                                                                 & D                \\ \hline
	\end{tabular}
	\caption{Adecuación}
	\label{tab:adecuacion}
\end{table}

\begin{table}[h]
	\centering
	\begin{tabular}{|l|l|l|l|p{4cm}|l|}
		\hline
		\scriptsize CATEGORÍA & \scriptsize IDENTIFICADOR & \scriptsize PESO & \scriptsize VALOR & \scriptsize DENOMINACIÓN DE LA CARACTERÍSTICA                                                                        & \scriptsize TIPO \\ \hline
		EX                    & E1                        & 8                & 10                & No se sienten amenazados por el proyecto, son capaces de sentirse intelectualmente unidos al proyecto                & D                \\ \hline
		EX                    & E2                        & 6                & 8                 & Tienen un brillante historial en la realización de esta tarea                                                        & D                \\ \hline
		EX                    & E3                        & 5                & 6                 & Hay acuerdos en lo que constituye una buena solución a la tarea                                                      & D                \\ \hline
		EX                    & E4                        & 5                & 8                 & La única justificación para dar un paso en la solución es la calidad de la solución final                            & D                \\ \hline
		EX                    & E5                        & 6                & 10                & No hay un plazo de finalización estricto, ni ningún otro proyecto depende de esta tarea                              & D                \\ \hline
		TA                    & E6                        & 7                & 8                 & No está influenciada por vaivenes políticos                                                                          & E                \\ \hline
		TA                    & E7                        & 8                & 10                & Existen ya SS.EE. que resuelvan esa o parecidas tareas                                                               & D                \\ \hline
		TA                    & E8                        & 8                & 10                & Hay cambios mínimos en los procedimientos habituales                                                                 & D                \\ \hline
		TA                    & E9                        & 5                & 10                & Las soluciones son explicables o interactivas                                                                        & D                \\ \hline
	\end{tabular}
	\caption{Éxito}
	\label{tab:exito1}
\end{table}
\begin{table}[h]
	\centering
	\begin{tabular}{|l|l|l|l|p{4cm}|l|}
		\hline
		\scriptsize CATEGORÍA & \scriptsize IDENTIFICADOR & \scriptsize PESO & \scriptsize VALOR & \scriptsize DENOMINACIÓN DE LA CARACTERÍSTICA                                                                        & \scriptsize TIPO \\ \hline
		TA                    & E10                       & 7                & 10                & La tarea es de I+D de carácter práctico, pero no ambas cosas simultáneamente                                         & E                \\ \hline
		DU                    & E11                       & 6                & 10                & Están mentalizados y tienen expectativas realistas tanto en el alcance como en las limitaciones                      & D                \\ \hline
		DU                    & E12                       & 7                & 10                & No rechazan de plano esta tecnología                                                                                 & E                \\ \hline
		DU                    & E13                       & 6                & 8                 & El sistema interactúa inteligente y amistosamente con el usuario                                                     & D                \\ \hline
		DU                    & E14                       & 9                & 8                 & El sistema es capaz de explicar al usuario su razonamiento                                                           & D                \\ \hline
		DU                    & E15                       & 8                & 10                & La inserción del sistema se efectúa sin traumas; es decir, apenas se interfiere en la rutina cotidiana de la empresa & D                \\ \hline
		DU                    & E16                       & 6                & 10                & Están comprometidos durante toda la duración del proyecto, incluso después de su implantación                        & D                \\ \hline
		DU                    & E17                       & 8                & 10                & Se efectúa una adecuada transferencia tecnológica                                                                    & E                \\ \hline
	\end{tabular}
\caption{Éxito -- continuación}
\label{tab:exito2}
\end{table}

%\subsection{Justificación}

\end{document}
